% THIS IS A LATEX TEMPLATE FILE FOR PAPERS INCLUDED IN THE
% *Anthology of Computers and the Humanities*. ADD THE OPTION
% 'final' WHEN CREATING THE FINAL VERSION OF THE PAPER. 
% DO NOT change the documentclass
\documentclass[final]{anthology-ch} % for the final version
%\documentclass{anthology-ch}         % for the submission

% LOAD LaTeX PACKAGES
\usepackage{booktabs}
\usepackage{graphicx}
% ADD your own packages using \usepackage{}

%\usepackage[greek]{babel}

\usepackage{booktabs}
\usepackage{tabularx}
\usepackage{listings}
\usepackage{comment}
\usepackage{verbatim}
\usepackage{svg}
\usepackage{subfig}
\usepackage{adjustbox}
\usepackage{multirow}

% Monospace font with Greek support

\newcommand{\gr}[1]{{\selectlanguage{greek}\greekfont #1}} % Greek text command

\newcommand{\detailtexcount}[1]{%
  \immediate\write18{texcount -merge -sum -q #1.tex output.bbl > #1.wcdetail }%
  \verbatiminput{#1.wcdetail}%
}

\newcommand{\quickwordcount}[1]{%
  \immediate\write18{texcount -1 -sum -merge -q #1.tex output.bbl > #1-words.sum }%
  \input{#1-words.sum} words%
}

\newcommand{\quickcharcount}[1]{%
  \immediate\write18{texcount -1 -sum -merge -char -q #1.tex output.bbl > #1-chars.sum }%
  \input{#1-chars.sum} characters (not including spaces)%
  }

\lstset{
  inputencoding=utf8,
  showstringspaces=true,
  extendedchars=true,
  basicstyle=\ttfamily,
  breaklines=true,
  escapechar=\%
% Wraps long lines
}


% TITLE OF THE SUBMISSION
% Change this to the name of your submission
\title{Echoes of Antiquity: Towards Understanding History through Human and LLM-Based Classical Text Translations}

% AUTHOR AND AFFILIATION INFORMATION
% For each author, include a new call to the \author command, with
% the numbers in brackets indicating the associated affiliations 
% (next section) and ORCID-ID for each author.  
\author[1]{Phillip Benjamin Ströbel}[
  orcid=0000-0003-2063-5495
]

\author[1]{Felix Klaus Maier}[
  orcid=0000-0002-5578-723X
]

% While we encourage including ORCID-IDs for all authors, you can
% include authors that do not have one by definining an empty ID.

% There should be one call to \affiliation for each affiliation of
% the authors. Multiple affiliations can be given to each author
% and an affiliation can be given to multiple authors. 
\affiliation{1}{Department of History, University of Zurich, Zurich, Switzerland}

% KEYWORDS
% Provide one or more keywords or key phrases seperated by commas
% using the following command
\keywords{translation studies, Ancient Greek, large language models, digital humanities, digital history}

% METADATA FOR THE PUBLICATION
% This will be filled in when the document is published; the values can
% be kept as their defaults when the file is submitted
\pubyear{2025}
\pubvolume{3}
\pagestart{44}
\pageend{44}
\conferencename{Computational Humanities Research 2025}
\conferenceeditors{Taylor Arnold, Margherita Fantoli, and Ruben Ros}
\doi{10.63744/XcjZ0MxpjIPj}  
\paperorder{4} 

\addbibresource{bibliography.bib}

%%%%%%%%%%%%%%%%%%%%%%%%%%%%%%%%%%%%%%%%%%%%%%%%%%%%%%%%%%%%%%%%%%%%%%%%%%%
% HERE IS THE START OF THE TEXT
\begin{document}

\maketitle

\begin{abstract}
This paper presents a systematic, data-driven comparison of human and large language model (LLM) translations of Ancient Greek texts, focusing on historiography and epic poetry. We assemble a parallel corpus of Greek source texts and multiple English translations, including both established human translations and new outputs from state-of-the-art LLMs such as GPT-4o, Gemini 1.5 Pro, and Claude 3.5 Sonnet. Using a unified computational workflow, we evaluate translations via lexical diversity, part-of-speech distributions, collocation profiles, and automatic translation quality metrics (BLEU, ROUGE, METEOR, chrF++). Our results reveal clear genre differences and demonstrate that, while LLMs can approach the fidelity of certain human translators and often reproduce dominant translation patterns, human versions retain greater interpretive diversity and linguistic nuance, especially in poetry. Collocation analyses further show that LLM outputs tend to converge on frequent patterns found in their training data, whereas human translators exhibit both shared conventions and unique, creative solutions. This work-in-progress study highlights both the potential and the present limitations of LLM-driven translation for classical scholarship, providing publicly available materials and quantitative benchmarks for future research in digital classics and translation studies.
\end{abstract}

\end{document}

\section{Introduction}
Modern historian Włodzimierz Lengauer controversially concludes his essay by advocating \emph{not} translating ancient texts at all. He contends that ``translation gives no idea of the original'' because it is impossible to reconstruct the meaning of texts written in languages and cultures that are now ``dead'' \parencite[pp.~26--27]{lengauer_traduttore_2013}. While this claim has sparked debate among scholars, it is rooted in the genuine complexities involved in rendering Ancient Greek into modern languages.\footnote{While Lengauer's examples focus on Polish, similar obstacles arise in translations from Ancient Greek to any modern language. This study focuses on Greek-to-English translation.}

Translators face several interconnected challenges: \textbf{(1) Poetic Language:} Ancient Greek poetry, as found in Homer’s epics, uses intricate metre and formulaic metaphors (e.g., \gr{ῥοδοδάκτυλος Ἠώς}, \textit{rosy-fingered dawn}). Conveying such poetic qualities while maintaining accuracy is non-trivial. \textbf{(2) Cultural Context:} Concepts like \gr{ἀρετή}, spanning \textit{excellence}, \textit{virtue}, and \textit{valour}, resist direct equivalence, requiring translators to bridge cultural gaps and provide adequate interpretation. \textbf{(3) Historical References:} Frequent allusions require contextual knowledge (e.g., understanding the Peloponnesian War in Thucydides). \textbf{(4) Ambiguity and Wordplay:} Ambiguities in the source often elicit diverse interpretive choices, challenging translation fidelity.

Lengauer’s ``non-translation'' solution is radical and, although intellectually rigorous, impractical, given the loss of access for most readers. We illustrate persistent translation challenges by a single sentence from Thucydides' \textit{Historiae} (Table~\ref{tab:example-translations}), where translators diverge on authorial voice and temporal phrases and none produces an identical outcome. This underscores that every translation involves choice and interpretation.

Such diversity is central in translation theory: Benjamin \cite{benjamin1972aufgabe} argues that translations should let the original ``echo'' in the target language, yet achieving perfect equivalence is impossible. Derrida \cite{derrida1987tours} extends this view, emphasising transformation and creative agency, while Jakobson \cite{jakobson1959linguistic} frames translation as interpretive recoding.

Despite these challenges, new translations continue to appear. Homer's \textit{Odyssey} alone has over 70 English versions~\cite{durantine_odyssey_translation_2018}, none of which are identical, highlighting how translators' backgrounds and exposure to previous versions influence the outcome~\cite{armstrong2005translating}. Translation thus becomes ``re-telling.''

Large Language Models (LLMs) are increasingly used for translation. Their outputs, shaped by training data and non-determinism, differ widely; e.g., Claude's output matches Cawley's, while GPT-4o and Gemini blend features from various translations (see Table~\ref{tab:example-translations-llms}). This research examines such variants across historiography and epic, probing how LLMs might replicate, recombine, or reinterpret ancient texts. At the end of this research (not this article), we aim to answer the question: What translation best captures the ancient mind?

%Modern historian Włodzimierz Lengauer controversially concludes his essay by advocating \emph{not} translating ancient texts at all. He contends that ``translation gives no idea of the original'' because it is impossible to reconstruct the meaning of texts written in languages and cultures that are now ``dead'' \parencite[pp.~26--27]{lengauer_traduttore_2013}. While this claim has sparked considerable debate among translators and scholars, it is rooted in the genuine complexities involved in rendering Ancient Greek into modern languages.\footnote{While Lengauer's examples focus on Polish, similar obstacles arise in translations from Ancient Greek to any modern language. This study focuses on Greek-to-English translation.}

%These challenges translators face include:


%\begin{itemize}
%    \item \textbf{Poetic Language}: Ancient Greek poetry, especially in Homer's epics, is characterised by intricate metrical patterns, formulaic expressions, and elaborate metaphors. E.g., consider the recurring epithet \gr{ῥοδοδάκτυλος Ἠώς} (EN \textit{rosy-fingered dawn}) used to describe the sunrise in the Odyssey. Translators must find ways to convey the poetic qualities of such expressions while maintaining accuracy and readability in the target language.
%    \item \textbf{Cultural Context}: Ancient Greek culture had distinct social norms, beliefs, and values that differed significantly from those of modern societies. Translators must bridge this cultural gap by providing clear explanations and accurate interpretations of culturally specific concepts, customs, and beliefs. E.g., the concept of \gr{ἀρετή} encompasses a range of meanings including \textit{excellence}, \textit{virtue}, and \textit{valour}, making it difficult to find a single English equivalent that fully captures its essence and thus requires careful consideration to ensure its meaning is accurately conveyed to a modern audience.
%    \item \textbf{Historical References}: Ancient Greek texts are filled with historical allusions and references to events, figures, and places that may be unfamiliar to modern readers (and even translators?). Translators must ensure that these references are accurately rendered and adequately explained. E.g., understanding the significance of the Peloponnesian War in Thucydides' work requires knowledge of the historical context and the key players involved.
%    \item \textbf{Ambiguity and Wordplay}: The Ancient Greek language often exhibits ambiguity and wordplay, which can be challenging to capture in translation. Translators must carefully consider the various possible interpretations and select the one that best aligns with the context and the author's intended meaning.
%\end{itemize}

%While Lengauer hints at the many challenges the translation processes pose, his proposed ``solution'', i.e., ``non-translation'' stands out for its originality: he contends that engagement with ancient texts should be reserved exclusively for those with exceptional linguistic mastery. While this bold stance carries a certain intellectual rigour, it is, in practice, unworkable, as it demands a level of expertise that is no longer attainable for many.

%We illustrate the persistent challenges of translation using a seemingly straightforward sentence from Thucydides' \textit{Historiae} (Table~\ref{tab:example-translations}). Notably, Cawley’s translation almost entirely omits the authorial first person, distancing the author's voice from the text, whereas Hammond, Mynott, and Warner each reference ``I'' or ``me'' at least three times. Further, the renditions differ in their portrayal of temporal distance, favouring phrases such as ``lapse of time,'' ``gap of time,'' ``length of time that has elapsed,'' and ``remoteness in time'' for the original phrase \gr{διὰ χρόνου πλήθος}. Thus, although each is based on the same Greek expression, none of the translations is identical to the others. This demonstrates that every translation involves choice and interpretation, even with uncontroversial source material.

%This diversity of outcomes is central to debates in translation theory. Walter Benjamin \cite{benjamin1972aufgabe} argues that the true aim of the translator is not merely to transmit information, but to let the original resound or ``echo'' in the target language by following its wording and structure as closely as possible. Yet, Benjamin concedes that fully capturing the original's meaning in a translation is unattainable; each translation inevitably falls short of the ``pure language'' of the source. Jacques Derrida \cite{derrida1987tours} extends this position: translation both preserves and transforms the original, endowing it with new meanings and ``afterlives.'' According to Derrida, translators are \emph{creative agents} shaped by their own linguistic and cultural milieu, whose activity always produces a difference, not transparency, between original and translation. This resonates with Roman Jakobson's influential model \cite{jakobson1959linguistic}, which views translation as a process of recoding, necessarily involving interpretation and restructuring rather than mere substitution.

%The example from Thucydides demonstrates how competing theories and practical realities converge: even the most literal segments of text give rise to a spectrum of valid translations, each embodying different translational philosophies and highlighting the creative, interpretive nature of the translator's task.

%TC:ignore
\begin{table}[]
\tiny
\begin{tabularx}{\textwidth}{X*{5}{>{\centering\arraybackslash}X}}
\toprule
\textbf{Original} & \textbf{Cawley} & \textbf{Hammond} & \textbf{Mynott} & \textbf{Warner}  \\ \midrule
\gr{τὰ γὰρ πρὸ αὐτῶν καὶ τὰ ἔτι παλαίτερα σαφῶς μὲν εὑρεῖν διὰ χρόνου πλῆθος ἀδύνατα ἦν, ἐκ δὲ τεκμηρίων ὧν ἐπὶ μακρότατον σκοποῦντί μοι πιστεῦσαι ξυμβαίνει οὐ μεγάλα νομίζω γενέσθαι οὔτε κατὰ τοὺς πολέμους οὔτε ἐς τὰ ἄλλα.} & For though the events of remote antiquity, and even those that more immediately precede the war, could not from lapse of time be clearly ascertained, yet the evidences which an inquiry carried as far back as was practicable leads me to trust, all point to the conclusion that there was nothing on a great scale, either in war or in other matters. & Accurate research into earlier or yet more ancient history was impossible given the great gap of time, but I have enquired as far into the past as I can, and on the evidence which I can trust I think there was nothing then on a large scale, either in wars or in anything else. & In respect of the preceding period and the still remoter past, the length of time that has elapsed made it impossible to ascertain clearly what happened; but from the evidence I find I can trust in pushing my enquiries back as far as possible, I judge that earlier events were not on the same scale, either as regards their wars or in other respects. & For though I have found it impossible, because of its remoteness in time, to acquire a really precise knowledge of the distant past or even of the history preceding our own period, yet, after looking back into it as far as I can, all the evidence leads me to conclude that these periods were not great periods either in warfare or in anything else. \\ \bottomrule
\end{tabularx}
\caption{Original and four translations of Thuc.~Hist.~I, 1,3.}
\label{tab:example-translations}
\end{table}
%TC:endignore

%in poetry: hexameter versus prose versus translation into an iambic metre \parencite{homer_odyssey_2018_wilson}

%Despite the difficulty of the translation task, a growing body of translations is being published. As of 2018, Homer's Odyssey alone counts over 70 published translations into English \cite{durantine_odyssey_translation_2018}. The constant publication of new translations demonstrates a high level of engagement with classical texts. We must assume that none of the 70 English translations of the Odyssey is 100\% identical, which shows that every translator contributes new interpretations of the source text. These interpretations depend on factors such as educational background, language proficiency and the ``shifting target-culture norms'' \cite{armstrong2005translating}. In this sense, the process of translating, or rather transposing from one language into another, is influenced by the translator's world views, which might resonate in the translation \cite{armstrong2005translating}. We further hypothesise that exposure to existing translations leads to a certain bias in each of the translators' work. As such, translating is also a form of re-composition and, as Richard Hamilton Armstrong puts it, ``re-telling'' \cite{armstrong2005translating}.


%Moreover, Large Language Models (LLMs) are increasingly being utilised in the translation process. Depending on the model used, the underlying training data and due to their non-deterministic nature, the results when prompted to provide a translation differ significantly. We provide examples for the same sentence already presented in Table \ref{tab:example-translations} from GPT-4o, Gemini 1.5 pro and Claude Sonnet 3.5 (see Section \ref{sec:methodology} for the model specificiations) in Table \ref{tab:example-translations-llms}. We see that Claude's translation is 1:1 copy from Cawley's, while the other two LLMs are ``mixtures'' of the other translations and exhibit some form of ``creativity.''  

%TC:ignore
\begin{table}[]
\tiny
\begin{tabularx}{\textwidth}{X*{4}{>{\centering\arraybackslash}X}}
\toprule
Original  & GPT-4o & Gemini 1.5 Pro  & Claude 3.5 Sonnet                               \\ \midrule
\gr{τὰ γὰρ πρὸ αὐτῶν καὶ τὰ ἔτι παλαίτερα σαφῶς μὲν εὑρεῖν διὰ χρόνου πλῆθος ἀδύνατα ἦν, ἐκ δὲ τεκμηρίων ὧν ἐπὶ μακρότατον σκοποῦντί μοι πιστεῦσαι ξυμβαίνει οὐ μεγάλα νομίζω γενέσθαι οὔτε κατὰ τοὺς πολέμους οὔτε ἐς τὰ ἄλλα.} & For it was impossible to find clearly the events before them and those even older due to the length of time, but from the evidence that I have examined as far back as possible, I conclude that they were not significant either in terms of wars or in other respects. & For the events of an earlier date, and those still more remote, the lapse of time makes it impossible to know with certainty; but from such evidence as I have been able to obtain by careful inquiry, I think that they were not very important, either in the wars themselves, or otherwise. & For though the events of remote antiquity, and even those that more immediately preceded the war, could not from lapse of time be clearly ascertained, yet the evidences which an inquiry carried as far back as was practicable leads me to trust, all point to the conclusion that there was nothing on a great scale, either in war or in other matters. \\ \bottomrule
\end{tabularx}
\caption{Translations of Thuc. Hist. I, 1,3 by GPT-4o, Gemini 1.5 Pro and Claude 3.5 Sonnet. }
\label{tab:example-translations-llms}
\end{table}
%TC:endignore

%But what exactly are the differences in the translations? This article aims to examine translations of classical literature. We focus on two genres, namely historiography and epic poetry. By incorporating LLM-based translation, we gain deeper insights into what might be in the training data of these LLMs. Moreover, a deeper analysis of such results will show us how we can employ LLMs in the translation of ancient texts and how (and if) they allow for re-interpretations of the ancient sources. At the end of this research (not this article), we aim to answer the question: What translation captures the ancient mind best?

\subsection{Our contribution}
This article presents work-in-progress for systematically comparing and aligning multiple translations of Ancient Greek, combining human editions with outputs from state-of-the-art LLMs. Our aim is (1) to assemble a parallel corpus that enables like-for-like analysis across genres (historiography and epic) and translator traditions, and (2) to develop a measurement toolkit (see Section \ref{sec:textprocandstat}) to quantify where renderings converge or diverge.

Conceptually, we treat translation as a measurable act of historical interpretation. Differences between renderings are not noise but signals, i.e., operational traces of reception that reveal how ancient conceptual categories are preserved, shifted, or re-weighted in modern language. Hence, we approximate what we call the ``echo'' of the ancient mind by the set of semantic and stylistic constraints that persist across translators and systems. The quantitative results, therefore, do not replace close reading; they prioritise it by flagging loci of interpretive pressure where qualitative analysis is most informative.

This paper contributes three elements toward that broader programme: \textbf{(i)} a reproducible corpus and processing pipeline; \textbf{(ii)} a comparative battery of metrics sensitive to genre and style; and \textbf{(iii)} preliminary evidence that LLM translations cluster around specific human exemplars, suggesting training data imprint or stylistic convergence. 

Looking ahead, we will extend the corpus (languages, genres), test memorisation versus generalisation, and formalise an ``interpretive distance'' index that integrates semantic similarity with controlled concept inventories (e.g., ritual, polity, kinship). The goal is not to declare a single best translation, but to characterise how interpretive space is shaped. Following Martindale’s reception-oriented view, we treat translation as a historically situated act of interpretation \cite{martindale1993redeeming}.

\begin{comment}
\section{Related Work}
\label{sec:relatedwork}

Sun and Li~\cite{sun2020dhtotranslation} review the significant impact that digital humanities (DH) methodologies have had on literary translation research in recent decades. They emphasise that, whereas earlier studies of translation often relied on subjective interpretation of limited examples, the advent of corpus-based and computational approaches has enabled the large-scale, systematic study of translated texts. Research in this tradition, building on Mona Baker's pioneering work \cite{baker1993cltranslation}, has introduced empirical methods for identifying regularities in translation (such as explicitation, simplification, and normalisation) as well as the distinctive statistical patterns that may characterise individual translators’ styles. Sun and Li further highlight how developments from monolingual comparable corpora to parallel and multiple-corpus models have refined the capacity to compare translation choices directly and to investigate the influence of both the source and target language context. Importantly, they note that advances in digital tools, i.e., text mining, data visualisation, and network analysis, have expanded the analytical toolkit for translation studies, permitting both macro-level surveys of translation activity and micro-level analyses of stylistic and discourse features across large datasets. Crucially, Sun and Li argue for the complementarity of computational analysis and traditional close reading, urging attention to both statistical regularities and contextual interpretation. Their survey substantiates the need for systematic, data-driven approaches to identify and explain the diversity observed in translations of the same text, providing essential methodological foundations for our comparative examination of classical and LLM-based translation outputs.

\subsection{Automatic Translation of Ancient Texts}

Recent work has established LLMs as the state of the art for translating ancient languages, surpassing earlier neural and rule-based approaches in both quality and flexibility. Wannaz and Miyagawa~\parencite{wannaz-miyagawa-2024-assessing} demonstrate that top LLMs such as GPT-4o, Claude Opus, and custom domain-specific models (notably the Coptic Translator) are now able to produce English translations of Ancient Greek and Coptic ostraca that approach the reliability and accuracy of expert human translators for certain text genres. Their systematic benchmarking using six NLP metrics reveals that current models perform particularly well on Ancient Greek owing to larger and higher quality training corpora, while results for Coptic remain more uneven, highlighting the persistent challenge posed by low-resource languages. Importantly, the study emphasises that LLM translation quality is not uniform: hallucination rates, translation consistency, and sensitivity to source corpus size and genre must be critically evaluated in each context.

Chamali \cite{chamali2025greek} investigates the persistent challenges that even state-of-the-art LLMs and neural translation models face in dealing with informal and culturally embedded language, specifically Greek slang and idioms. By constructing novel parallel datasets for both phenomena and evaluating three leading LLMs (Gemma, Llama, Mistral) as well as a specialist NMT model, she demonstrates that translation quality remains consistently poor, regardless of model type. Notably, her error analysis reveals substantial difficulties in informativeness and a wide variety of failure types, underscoring that culturally specific and colloquial expressions remain a major obstacle for automatic translation—especially for under-resourced and morphologically rich languages such as Greek.


Tekgürler~\cite{tekgurler2025llmstranslationhistoricallowresourced} expands this perspective by demonstrating how state-of-the-art LLMs like Google Gemini can tackle the translation of 18th-century Ottoman Turkish manuscripts, but also highlights the new challenge posed by AI content moderation. Her results show that automatic safety measures may redact or alter essential passages, specially those concerning violence and sensitive historical events, thus raising critical issues about the completeness and fidelity of AI-assisted historical translation.

Further evidence for the efficacy of LLMs in this field comes from recent case studies on other ancient and historical languages. Volk et al.~\cite{volk-etal-2024-llm} evaluates the translation capability of GPT-4 and Gemini on 16\textsuperscript{th}-century Latin letters, reporting that LLM-generated translations and cross-language summaries are not only superior to previous machine translation systems but often nearly as usable as professionally published human editions. Dedicated open-source models such as Krikri~\cite{roussis2025krikriadvancingopenlarge} have advanced the translation of Ancient into Modern Greek, with a BLEU score of 54.66. These successes are enabled by careful fine-tuning, targeted preprocessing, and the development of evaluation benchmarks specifically adapted for historical texts.

Concurrently, research in digital humanities is exploiting LLMs to build comparative translation databases and analytical pipelines for Greek, Latin, and additional classical languages \cite{voukoutis2024meltemiopenlargelanguage}. These initiatives use LLMs alongside human and legacy machine translations to map stylistic, grammatical, and semantic variation across centuries and multiple target languages. Projects in this space have begun to integrate retrieval-augmented generation (RAG)\footnote{See \textcite{gao2024retrievalaugmentedgenerationlargelanguage} for an overview.} and hybrid methodologies~\cite{miyagawa-2025-rag}, further improving translation performance and enabling robust applications such as digital reading assistants and interactive annotation tools for ancient texts. Despite these advances, all studies underscore the ongoing necessity of expert human oversight to validate and curate LLM-generated translations, especially given the risks associated with bias, omission, and the handling of fragmentary or culturally sensitive content.

\subsection{Translation Comparisons}
Translation comparison studies have gained prominence in recent years through the application of digital methods and the widespread use of parallel corpora. Scholars increasingly align multiple translations of a single source text to systematically investigate both lexical variation and broader interpretive trends. For example, Chiara Palladino et al.~\parencite{palladino2022using, Palladino-2023} present a workflow that combines manual word-level alignment with computational analysis to compare translations of Ancient Greek texts into English and Persian. Their methodology quantifies overlaps and differences using word-level intersections, part-of-speech correspondences, and alignment types, revealing how individual translators handle challenges such as expansion, omission, or adaptation, even in passages that might appear straightforward. Notably, their findings indicate that translations made directly from the source language tend to prioritise accuracy, while those mediated through a third language exhibit increased variability, and that a combination of quantitative and qualitative criteria can yield insights unavailable through close reading alone.

Parallel corpora have, as a result, become a standard resource for translation comparison and the study of variation. Michael Barlow \cite{barlow-2004-parallel} and other scholars have shown that tools like parallel concordancers and alignment software facilitate precise comparisons between pairs or sets of translations, allowing researchers to isolate specific translational choices, trace the consistency of terminology, and pinpoint divergence or creative interpretation at the level of individual sentences or phrases. These studies not only enhance empirical rigour in translation research but also inform translator training and evaluation. Furthermore, contemporary comparison projects often consider both direct and indirect translations, address distinctive translator styles, and analyse strategies across different genres and historical periods, thus advancing the field from impressionistic comparisons to replicable, data-driven accounts of translation practice.
\end{comment}

\section{Related Work}
\label{sec:relatedwork}

Sun and Li~\cite{sun2020dhtotranslation} comprehensively chart the transformation of literary translation research via digital humanities (DH) methodologies. They argue that the integration of corpus-based and computational approaches replaced earlier, subjective paradigms with systematic and large-scale investigations. Drawing on foundational work by Baker~\cite{baker1993cltranslation}, Sun and Li detail how empirically defined phenomena (such as explicitation, simplification, and normalisation) are now observable and quantifiable across parallel and comparable corpora. These developments, aided by advances in text mining, visualisation, and network analysis, have allowed researchers to detect both macro-level trends in translation activity and micro-level stylistic ``thumbprints'' of individual translators. Their review stresses the value of combining close reading with computational analysis in order to interpret observed variation, establishing a methodological baseline for our comparative work on classical and LLM-generated translations.

\subsection{Automatic Translation of Ancient Texts}

Recent advances in LLMs have redefined the state-of-the-art in ancient language translation. Wannaz and Miyagawa~\parencite{wannaz-miyagawa-2024-assessing} benchmarked leading LLMs, including GPT-4o, Claude Opus, and a domain-specific Coptic Translator, showing that these models can generate English translations for Ancient Greek and Coptic \textit{ostraca} that approach expert-level reliability, especially for Greek due to richer training resources. However, their findings emphasise disparities between languages, performance issues for low-resourced cases, and differences in hallucination rates, translation consistency, and genre sensitivity.

Chamali~\parencite{chamali2025greek} investigates challenges even for state-of-the-art models by analysing translations of Modern Greek slang and idioms. By creating parallel datasets and evaluating LLMs alongside an NMT baseline, she finds that model performance remains weak for culturally embedded and colloquial language, with diverse error types and informativeness failures, demonstrating the enduring limits of LLMs for morphologically rich and under-resourced languages.

Tekgürler~\parencite{tekgurler2025llmstranslationhistoricallowresourced} broadens the scope, showing how LLMs approach the translation of 18\textsuperscript{th}-century Ottoman Turkish but also highlighting a new set of challenges: contemporary AI content moderation can redact or distort source material, especially with references to violence or sensitive events, raising concerns about the completeness and reliability of automatic historical translation.

Further validation comes from Volk et al.~\parencite{volk-etal-2024-llm}, who show that LLMs can produce translations and summaries of 16\textsuperscript{th}-century Latin texts close in quality to published human equivalents. Additionally, the Krikri model~\cite{roussis2025krikriadvancingopenlarge} has significantly advanced Ancient-to-Modern Greek translation, achieving high benchmark scores through fine-tuning and robust pre-processing. Research increasingly uses LLMs in digital humanities pipelines and comparative translation databases across Greek, Latin, and related languages~\cite{voukoutis2024meltemiopenlargelanguage}. Projects in this space combine LLMs, retrieval-augmented generation ~\cite{miyagawa-2025-rag}, and hybrid methodologies, supporting digital annotation tools, reading assistance, and deeper stylistic or semantic mapping. The literature consistently highlights the ongoing importance of expert review in validating model outputs and addressing risks such as bias, omissions, and issues with fragmentary or culturally complex historical texts.

\subsection{Translation Comparisons}

Translation comparison studies have gained momentum thanks to digital methods and the increased use of parallel corpora. Palladino et al.~\parencite{palladino2022using, Palladino-2023} present an approach that blends manual word-level alignment with computational metrics to assess translation variation between English and Persian renditions of Ancient Greek sources. Their methodology quantifies overlaps, part-of-speech (POS) correspondence, and alignment categories, enabling precise tracking of where translators choose expansion, omission, or adaptation and showing that indirect (e.g., non-source-language) translations are especially variable.

Parallel corpora and digital concordancers have consequently become central to translation comparison and stylistic analysis. Barlow~\cite{barlow-2004-parallel} and others demonstrate how specialised software supports fine-grained comparison of multiple translations, making translator choices, terminology consistency, and stylistic divergence traceable at sentence- and phrase-level. Such studies not only establish rigorous empirical standards for translation research but also contribute to translator education and training. The latest projects expand the scope to direct and indirect translations, systematically examine individual translator styles, and analyse strategies across diverse genres and historical settings, moving the field towards transparent, replicable accounts of translation variation and choice.

\begin{comment}
\section{Data}
\label{sec:data}
In what follows, we introduce the translations\footnote{The year next to the translator's name indicates when the translation was first published. We specify in the running text which version (or reprint) we used.} we used and want to analyse with the help of LLMs (see Section \ref{sec:methodology}) from oldest to newest. We take details characterising the translations directly from the respective translators' introductions or notes on the text, since each one attempted to describe their result and justify some decisions they made during the translation process.

\subsection{Ancient Texts and Their Translations}
\label{sec:sources}

\subsubsection{Homer's \textit{Odyssey}}
\label{sec:odyssey}

\paragraph{The Odyssey} We used the edition \textcite{homer}.

\paragraph{Robert Fitzgerald, 1961} We used the edition \parencite{homer_odyssey_1998_fitzgerald}

\paragraph{Richmond Lattimore, 1965} The translation by XYZ Richmod Lattimore \parencite{homer_odyssey_1967_lattimore}

\paragraph{Robert Fagles, 1996} Robert Fagle's translation \parencite{homer_odyssey_1996_fagles}

\paragraph{Ahthony S. Kline, 2004} During the translation of the Odyssey, Anthony S. Kline focused on ... \parencite{homer_odyssey_2004_kline}

\paragraph{Emily Wilson, 2017} A pecularity of Emily Wilson's translation \parencite{homer_odyssey_2018_wilson} is that ...

\paragraph{Peter Green, 2018} The translation by Peter Green \parencite{homer_odyssey_1998_green} ...

\subsubsection{Thucydides' \textit{Historiae} or \textit{History of the Peloponnesian War}}
\label{sec:thucydides}

\paragraph{Historiae} We used the edition \textcite{thucydides}

\paragraph{Charles Forster Smith, 1919} The oldest translation in our experimental setup is by Charles Forster Smith \parencite{thuc_historiae_1919_smith}
    
\paragraph{Rex Warner, 1954} In contrast to X and Z, Rex Warner opted for ... in his translation \parencite{thuc_historiae_1954_warner}.

\paragraph{Martin Hammond, 2009} In his translation \parencite{thuc_historiae_2009_hammond}, Martin Hammond took great care to ....

\paragraph{Jeremy Mynott, 2013} Jeremy Mynott's translation is especially ... \parencite{thuc_historiae_2013_mynott}
\end{comment}
\section{Data}
\label{sec:data}

We introduce the edition of the source texts and the translations\footnote{The year next to the translator's name indicates when the translation was first published.} used in this study, which are analysed with the help of LLMs (see Section~\ref{sec:methodology}). Rather than detailing each translation individually,\footnote{Key characteristics can be drawn from translators' prefaces and notes.} we provide an overview in tabular form, summarising the major editions of Homer’s \textit{Odyssey} and Thucydides’ \textit{Historiae} across different periods and approaches.

%The selection of the translation was mainly at random, with the exception of Wilson's \cite{wilson_odyssey_2018}, which i 
The selected translations span a broad range of publication dates, stylistic approaches, and translational philosophies. Differences arise in the translators' degree of literalness, their handling of narrative voice, and their strategies for rendering cultural concepts: some editions emphasise fidelity to metrical structure or archaic diction (e.g., Wilson's \cite{wilson_odyssey_2018}, which provides a conversion from the dactylic hexameter to the iambic pentameter, which is more ``natural'' to English), while others favour accessibility and contemporary language. Such choices lead to variation in the treatment of poetic epithets, authorial interventions, and culturally embedded terms, yielding distinct interpretive outcomes from the same Greek source material.

%TC:ignore
\begin{table}[ht]
\centering
\begin{tabularx}{\textwidth}{X X}
\toprule
\textbf{Homer's \textit{Odyssey}} \parencite{homer} & \textbf{Thucydides' \textit{Historiae}} \parencite{thucydides} \\
\midrule
Robert Fitzgerald (1961) \parencite{homer_odyssey_1998_fitzgerald} & Richard Cawley (1914) \parencite{thuc_historiae_1914_cawley} \\
Richmond Lattimore (1965) \parencite{homer_odyssey_1967_lattimore} & Rex Warner (1954) \parencite{thuc_historiae_1954_warner} \\
Robert Fagles (1996) \parencite{homer_odyssey_1996_fagles} & Martin Hammond (2009) \parencite{thuc_historiae_2009_hammond} \\
Anthony S. Kline (2004) \parencite{homer_odyssey_2004_kline} & Jeremy Mynott (2013) \parencite{thuc_historiae_2013_mynott} \\
Emily Wilson (2017) \parencite{homer_odyssey_2018_wilson} &  \\
Peter Green (2018) \parencite{homer_odyssey_1998_green} & \\
\bottomrule
\end{tabularx}
\caption{Summary of selected English translations analysed for Homer's \textit{Odyssey} and Thucydides' \textit{Historiae}. Publication years correspond to first editions.}
\label{tab:translations}
\end{table}
%TC:endignore

\section{Methodology}
\label{sec:methodology}

Our methodology encompasses a comparative analysis of English translations of Ancient Greek texts, produced by both human translators and LLMs. Our workflow ensures consistency, replicability, and a quantitative foundation for assessing stylistic, lexical, and syntactic features. All the data and the scripts are available on GitHub.\footnote{See \url{https://github.com/AncientHistory-UZH/echos-of-antiquity}.}

\subsection{Source Texts and Translations}
\begin{itemize}
  \item \textbf{Ancient Greek Sources:} We conducted linguistic analyses of Greek source texts (sentence splitting, tokenisation, POS tagging, lemmatisation) using \texttt{odyCy} \cite{kostkan_odycy_2023} and saved the output in the CoNLL format. 
  \item \textbf{English Translations:} Both human- and LLM-generated translations (see below) underwent the same preprocessing steps as the Ancient Greek sources, albeit using a different spaCy model. We used the English model \texttt{en\_core\_web\_md} and saved the result in CoNLL format.
\end{itemize}

\subsection{Large Language Models}
\label{sec:llms}
For translation, we employed three state-of-the-art models: (1) OpenAI's \textbf{GPT-4o} (\texttt{gpt-4o}), (2) Google's \textbf{Gemini 1.5 Pro} (\texttt{gemini-1.5-pro}), and (3) Anthropic's \textbf{Claude 3.5 Sonnet} (\texttt{claude-3\\-5-sonnet-20240620}). We used corresponding APIs, batch processing and the same prompts overall to ensure comparability. Firstly, we prompted all LLMs\footnote{See Appendix \ref{appdx:first}.\ref{lst:prompt-to-translate} for the prompt.} to translate the source texts, yielding three additional translations per LLM.

Next, we used Anthropic's Claude 3.5 Sonnet (\texttt{claude-3-5-sonnet-20241022}) to align the source and target texts, including the three additional automatic translations.\footnote{See Appendix \ref{appdx:first}.\ref{lst:prompt-to-align} for the prompt.} We used the alignment to compute the cosine similarities between source and target and target and target sentences (see below).

%This resulted in 9 alignments for the \textit{Odyssey} and 7 alignments for the \textit{Historiae}. We plan to extend the alignments also to GPT-4o and Gemini 1.5 Pro in future work. In general, an evaluation of the alignment quality will indicate how easily LLMs can establish a connection between the source and the target. We hypothesise that the alignments are worse for ``removed,'' i.e., non-literal translations, which would indicate more interpretative efforts from the translator. However, to have a comparison candidate, we employed \textit{Bertalign} \cite{liu2022bertalign}. Bertalign embeds source and target sentences using sentence Transformers (in this case, the LaBSE\footnote{See \url{https://huggingface.co/sentence-transformers/LaBSE}.} \cite{feng2022languageagnosticbertsentenceembedding} model), followed by a dynamic programming algorithm to first identify 1:1 alignments as anchor points to use them to infer the 1-to-many and many-to-1 alignments in a second step.\footnote{At this point in time, we cannot provide an evaluation of the alignment yet. However, we used it to compute the cosine similarities between source and target segments. Please note that the results are subject to change.}

\subsection{Text Processing and Statistical Analysis}
\label{sec:textprocandstat}

\subsubsection{Collocation Analysis}
\begin{itemize}
  \item \textbf{Tooling:} We used the NLTK\footnote{See \url{https://www.nltk.org}.} library to identify and rank frequent bigrams and trigrams.
  \item \textbf{Statistical Association:} We applied pointwise mutual information (PMI), chi-squared, student-t, log-likelihood and frequency counts (with frequency thresholds) to determine salient collocations for each translation.
  \item \textbf{Comparison:} We compared rankings and collocate lists (among the first 1000 collocations each) across LLM and human outputs to assess overlap, divergence, and distinctive phraseological patterns. We used Spearman's $\rho$ and Kendall's $\tau$ to assess differences in the translations. 
\end{itemize}

\subsubsection{Translation Comparison Measures}
\label{sec:trans-comp-measures}
To quantitatively evaluate the quality and similarity of LLM-generated translations against reference human translations, we employed a set of automatic metrics from the machine translation literature (see Table \ref{tab:translation-metrics}).

\begin{table}[ht!]
\centering
\begin{tabular}{p{0.22\linewidth} p{0.70\linewidth}}
\hline
\textbf{Metric} & \textbf{Description} \\
\hline
Cosine similarity & Measures the distance between embedded source and target sentences using LaBSE\footnote{\url{https://huggingface.co/sentence-transformers/LaBSE}} \cite{feng2022languageagnosticbertsentenceembedding}. \\
BLEU \cite{papineni-etal-2002-bleu} & Measures \textit{n}-gram precision between candidate and reference translations, with brevity penalties to control for length differences. \\
ROUGE \cite{lin-2004-rouge} & Focuses on recall-based overlap of \textit{n}-grams, suitable for evaluating content coverage. \\
METEOR \cite{banerjee-lavie-2005-meteor} & Incorporates both precision and recall, including stemming, synonymy, and paraphrase matching to capture semantic similarity better. \\
chrF++ \cite{popovic-2017-chrf} & A character \textit{n}-gram based metric combining F-scores for both precision and recall; well-suited for morphologically rich languages and sensitive to minor word-form variation. \\
\hline
\end{tabular}
\caption{Automatic translation evaluation metrics used in this study.}
\label{tab:translation-metrics}
\end{table}


%These complementary metrics allow for robust cross-evaluation of lexical, syntactic, and semantic aspects of translation output, providing a nuanced profile of LLM translations' fidelity and fluency relative to expert human benchmarks.

%\subsection{Workflow and Reproducibility}
%\begin{itemize}
%  \item All preprocessing scripts, statistical analyses, and comparison routines were documented in Jupyter notebooks (see supplementary material), ensuring transparency and reproducibility.
%  \item The workflow supports both corpus-level and passage-level analyses, allowing granular evaluation of translation strategies, stylistic tendencies, and linguistic profiles for Ancient Greek and English translations.
%\end{itemize}


\section{Preliminary Results and Analyses}
\label{sec:experiments}

\subsection{Statistical Analysis of Translations}
\label{sec:stats}

\subsubsection{Lexical Analysis}
\label{sec:lexicon}

\begin{figure}%
\centering
\subfloat[Lexical diversity scores of Homer's \textit{Odyssey} and its translations.]{\includesvg[width=0.46\textwidth]{images/homer-lexdiv-2.svg}}\qquad
\subfloat[Lexical diversity scores of Thucydides' \textit{Historiae} and its translations.]{\includesvg[width=0.46\textwidth]{images/thuc-lexdiv-2.svg}}\qquad
\caption{Lexical diversity (Type–Token Ratio) by book for each source and translation. \textbf{x-axis:} Book. \textbf{y-axis:} Type–Token Ratio (types/tokens). (Scales differ for Homer vs. Thucydides.)}
\label{fig:lexdiv-profiles}
\end{figure}

Figure \ref{fig:lexdiv-profiles} shows the Type-Token Ratio (TTR) of the two source texts and their translations. We notice a strong difference between the Ancient Greek sources: Homer's average TTR across all books is 0.31, while Thucydides' is 0.16, indicating Homer's greater vocabulary variation. For the translations, we note that the automatically generated translations are always at the bottom. This suggests either a more consistent translation of specific phrases or a limited command of the vocabulary. Pairwise t-tests between human and automatic translation show that the differences of the TTR of the \textit{Odyssey} between GPT-4o, Gemini, Claude Sonnet and Lattimore are not statistically significant. The t-tests for \textit{Historiae} show no significant differences whatsoever. Notably, the TTRs of different translations follow similar patterns (e.g., low in Book 3, high in Book 7), with some peaks and dips also aligning with the source texts (e.g., Book 19).

\subsubsection{Part-of-Speech Tag Patterns}
\label{sec:pos}
The POS tag profiles in Figure \ref{fig:pos-profiles} again show a difference between the two genres, especially in the usage of nouns and verbs. Homer's language is distinctively more ``noun-heavy'' than Thucydides', while the latter's usage of verbs is much higher. 

\begin{figure}%
\centering
\subfloat[POS profile of Homer's \textit{Odyssey} and its translations.]{\includesvg[width=0.46\textwidth]{images/pos-homer-2.svg}}\qquad
\subfloat[POS profile of Thucydides' \textit{Historiae} and its translations.]{\includesvg[width=0.46\textwidth]{images/pos-thyc-2.svg}}\qquad
\caption{UPOS distribution profiles for source and translations (grouped bar charts). \textbf{x-axis}: UPOS tag. \textbf{y-axis}: Percentage of tokens.}
\label{fig:pos-profiles}
\end{figure}

%Manual steps: the outputs provided by Claude Sonnet needed some manual post-processing since the XML it returned was not always correct. There were 101 instances of tag mismatches

%Lattimore 8.516 is missing


\subsection{Collocations}
\label{sec:collocations}

\subsubsection{Automatic alignment}
\label{sec:automaticalignment}

For the preliminary collocation analysis, we refer to Appendix \ref{app:second}. Collocation profiles suggest that human translations, despite interpretive variation, often show higher intercorrelation in their phraseological patterns, particularly for established translators of the same era and literary orientation. The p-values of Kendall's $\tau$ analysis can indicate which LLM translations are closely related to specific reference translations. E.g., the trigrams identified in Thucydides with the help of the chi-square measure between GPT-4o and Cawley show a p-value of 0.007, suggesting that the commonality among the first 1000 trigrams in both translations does not occur by chance.  As such, some LLMs align more closely with their likely training exemplars, while others exhibit patterns distinct from those of any individual human translator. 

Concretely, in Table \ref{tab:example-translations-llms} (Thuc. I.1.3), Claude's rendering tracks Cawley's clause structure and connective choices unusually closely; GPT-4o and Gemini combine elements characteristic of Hammond and Mynott (e.g., treatment of ``remote antiquity'' vs. ``preceding period''). This pattern is consistent with our broader collocation–rank correlations, suggesting that LLMs preferentially reproduce entrenched translation routines. While this does not prove direct memorisation, it supports the hypothesis of stylistic imprint from widely circulated public-domain or widely quoted translations.

%Guided by spikes in our collocation divergences, we performed a brief close reading on Odyssey Book 7 and Thucydides I.1.3. We annotated instances of expansion (added explanatory phrasing), omission (compressed or dropped qualifiers), and domestication vs.~foreignisation (handling of culturally marked terms). Human translators exhibited both shared conventions and idiosyncratic solutions (e.g., variation in rendering epithets and temporal qualifiers), whereas LLM outputs tended to select high-frequency conventional solutions with fewer expansions and fewer culturally foreignising choices. This supports our quantitative finding that machine renderings converge on dominant patterns and preserve semantic adequacy, but show reduced interpretive breadth relative to human translations.



%Overall, the collocation statistics support the conclusion that while LLMs can reproduce the frequent phraseology typical of human-generated translations, the greatest interpretive variety (and, by extension, the richest translation choices) continues to originate from the human tradition. 

%\paragraph{Human Evaluation of Alignment on 100 Randomly Sampled Alignments}


\subsubsection{Evaluation of Translation}
\label{sec:automatictranslation}

\begin{figure}%
\centering
\subfloat[Distribution of translation fidelities of Homer's \textit{Odyssey}.]
{\includegraphics[scale=0.65]{images/fidelityHomer-Odyssey(Greek).pdf}}\qquad
\subfloat[Distribution of translation fidelities of Thucydides' \textit{Historiae}.]{\includegraphics[scale=0.65]{images/fidelityThucydides-History(Greek).pdf}}\qquad
\caption{Distribution of translation fidelities of the translations. On the x-axis, we see the range of the cosine similarities, while the y-axis shows the respective translations (not the difference in scale (x-axis)).}
\label{fig:trans-fidelity}
\end{figure}

Figure \ref{fig:trans-fidelity} shows the box plots of cosine similarity scores between each sentence of the source and target texts in the embedding space created by LaBSE. As with the TTR and POS tag profiles, we observe differences between Homer and Thucydides. Plots (a) and (b) also show that the similarities between the source and automatically generated translations are greater. Since LaBSE was trained on the Common Crawl corpus and Wikipedia, it is possible that the Ancient Greek source texts and translations in the public domain were in the training data.


% Please add the following required packages to your document preamble:
% \usepackage{multirow}
\begin{table}[]
\resizebox{\textwidth}{!}{%
\begin{tabular}{llrrrrr}
\hline
\textbf{Machine Translation}   & \textbf{Human Reference} & \multicolumn{1}{l}{\textbf{Semantic Similarity}} & \multicolumn{1}{l}{\textbf{BLEU}} & \multicolumn{1}{l}{\textbf{ROUGE}} & \multicolumn{1}{l}{\textbf{METEOR}} & \multicolumn{1}{l}{\textbf{chrF++}} \\ \hline
\multirow{6}{*}{Claude Sonnet} & Fagles                   & 0.576                                            & 0.060                             & 0.300                              & 0.279                               & 24.981                              \\
                               & Fitzgerald               & 0.511                                            & 0.055                             & 0.250                              & 0.250                               & 21.824                              \\
                               & Green                    & 0.685                                            & 0.099                             & 0.392                              & 0.383                               & 29.435                              \\
                               & Kline                    & 0.576                                            & 0.060                             & 0.300                              & 0.279                               & 24.981                              \\
                               & Lattimore                & 0.744                                            & 0.140                             & 0.445                              & 0.424                               & 35.963                              \\
                               & Wilson                   & 0.554                                            & 0.066                             & 0.296                              & 0.304                               & 24.884                              \\ \hline
\multirow{6}{*}{GPT-4o}        & Fagles                   & 0.562                                            & 0.052                             & 0.286                              & 0.265                               & 23.759                              \\
                               & Fitzgerald               & 0.500                                            & 0.045                             & 0.240                              & 0.239                               & 20.761                              \\
                               & Green                    & 0.666                                            & 0.083                             & 0.367                              & 0.355                               & 27.398                              \\
                               & Kline                    & 0.562                                            & 0.052                             & 0.286                              & 0.265                               & 23.759                              \\
                               & Lattimore                & 0.729                                            & 0.123                             & 0.426                              & 0.401                               & 34.000                              \\
                               & Wilson                   & 0.546                                            & 0.060                             & 0.289                              & 0.297                               & 24.127                              \\ \hline
\multirow{6}{*}{Gemini}        & Fagles                   & 0.567                                            & 0.054                             & 0.287                              & 0.268                               & 24.180                              \\
                               & Fitzgerald               & 0.504                                            & 0.049                             & 0.244                              & 0.243                               & 21.261                              \\
                               & Green                    & 0.674                                            & 0.089                             & 0.375                              & 0.366                               & 28.270                              \\
                               & Kline                    & 0.567                                            & 0.054                             & 0.287                              & 0.268                               & 24.180                              \\
                               & Lattimore                & 0.736                                            & 0.131                             & 0.430                              & 0.411                               & 34.911                              \\
                               & Wilson                   & 0.546                                            & 0.058                             & 0.285                              & 0.293                               & 24.091                              \\ \hline
\end{tabular}%
}
\caption{Translation measures of Homer's \textit{Odyssey}.}
\label{tab:trans-measures-odyssey}
\end{table}

% Please add the following required packages to your document preamble:
% \usepackage{booktabs}
% \usepackage{multirow}
\begin{table}[]
\resizebox{\textwidth}{!}{%
\begin{tabular}{@{}llrrrrr@{}}
\toprule
\textbf{Machine Translation}   & \textbf{Human Reference} & \multicolumn{1}{l}{\textbf{Semantic Similarity}} & \multicolumn{1}{l}{\textbf{BLEU}} & \multicolumn{1}{l}{\textbf{ROUGE}} & \multicolumn{1}{l}{\textbf{METEOR}} & \multicolumn{1}{l}{\textbf{chrF++}} \\ \midrule
\multirow{4}{*}{Claude Sonnet} & Cawley                   & 0.668                                            & 0.144                             & 0.374                              & 0.379                               & 35.73                               \\
                               & Hammond                  & 0.752                                            & 0.155                             & 0.416                              & 0.426                               & 40.14                               \\
                               & Mynott                   & 0.777                                            & 0.176                             & 0.456                              & 0.457                               & 42.198                              \\
                               & Warner                   & 0.755                                            & 0.152                             & 0.412                              & 0.408                               & 40.152                              \\ \midrule
\multirow{4}{*}{GPT-4o}        & Cawley                   & 0.646                                            & 0.092                             & 0.323                              & 0.322                               & 31.111                              \\
                               & Hammond                  & 0.729                                            & 0.118                             & 0.374                              & 0.373                               & 36.138                              \\
                               & Mynott                   & 0.756                                            & 0.129                             & 0.411                              & 0.398                               & 37.598                              \\
                               & Warner                   & 0.729                                            & 0.108                             & 0.363                              & 0.351                               & 35.34                               \\ \midrule
\multirow{4}{*}{Gemini}        & Cawley                   & 0.661                                            & 0.125                             & 0.349                              & 0.363                               & 34.598                              \\
                               & Hammond                  & 0.737                                            & 0.138                             & 0.387                              & 0.401                               & 38.653                              \\
                               & Mynott                   & 0.764                                            & 0.154                             & 0.423                              & 0.429                               & 40.297                              \\
                               & Warner                   & 0.74                                             & 0.136                             & 0.382                              & 0.385                               & 38.592                              \\ \bottomrule
\end{tabular}%
}
\caption{Translation measures for Thucydides' \textit{Historiae}.}
\label{tab:trans-measures-thuc}
\end{table}

Tables \ref{tab:trans-measures-odyssey} and \ref{tab:trans-measures-thuc} show the general evaluation according to the measures introduced in Section \ref{sec:trans-comp-measures}. The semantic similarity score indicates which human translation is closest to the automatic translation. For Homer, Lattimore's translation is, on average, the most similar to all the automatic translations. We must thus hypothesise that Lattimore's text has been included in the training data of all three LLMs. For Thucydides, this is the case for Hammond, Mynott, and Warner. The overall lower BLEU scores for the \textit{Odyssey} when compared to the \textit{Historiae} again emphasise the genre difference. It appears that translating poetry yields more diverse results. The remaining scores correlate well with the semantic similarity and BLEU scores.

As concerns further interpretation, we note that BLEU/ROUGE (\textit{n}-gram overlap) emphasise surface correspondence. In contrast, METEOR and chrF++ better tolerate paraphrase and morphological variation, and the sentence-embedding similarity foregrounds semantic proximity. The consistent proximity of LLM outputs to Lattimore (Homer) and to Hammond/Mynott/Warner (Thucydides) indicates convergence on dominant rendering patterns rather than uniformly higher ``quality.'' In other words, alignment seems stylistic-semantic: models mirror phrasing conventions that likely occur in pre-existing translations, while maintaining adequate semantic coverage. This explains why poetry (Odyssey) shows relatively lower overlap scores than prose (Historiae): meter and figurative density invite broader paraphrastic space, where models track conventional solutions but rarely expand the interpretive palette beyond them.



%\subsubsection{Word Translation Differences}
%\label{sec:wordchoice}

%\subsubsection{Semantic Similarities}

%\section{Results and Discussion}

\section{Conclusion}

%Nonetheless, while the importance of language proficiency cannot be overstated, the essence of revitalising classical studies lies in ensuring accessible pathways to antiquity. Only by lowering barriers can the fascination of the ancient world be shared with a broader audience, securing its relevance and appeal in a modern context.

%After all, texts like Thucydides' \textit{Historiae} or \textit{The Peloponnesian War} \parencite{thucydides} offer a window into past events, societies and cultures, albeit in a very different style than we are used to. And while historical sciences have long moved away from taking accounts in ancient texts at face value, they allow us to imagine the past in a way that can culminate in \textit{imitatio}, a theme that was all too present in the Renaissance \parencite[p.~8]{schiffman2011birth}.
Our findings demonstrate that both human experts and state-of-the-art LLMs produce translations of Ancient Greek that are systematically distinct in measurable ways, particularly in style, lexicon, and fidelity to the source. Lexical diversity analyses confirm significant differences between poetry and historiography, with Homer’s original demonstrating a vastly richer vocabulary than Thucydides, and these patterns are reliably reflected in both human and machine translations. POS distributions further reveal genre-specific syntactic tendencies and highlight the degree to which translators (and LLMs) reproduce or diverge from the source structure.

Automatic evaluation metrics indicate that LLM outputs are often closest to specific human translations (notably Lattimore for Homer), suggesting that the training data overlap or algorithmic convergence occurs on a dominant translation style. Yet, the lower scores for poetry versus prose also confirm that genre affects translation diversity and model performance. Our results suggest that while LLMs can approximate established translation choices, they may reinforce conventional renderings rather than innovate or meaningfully expand interpretive space. Finally, the combined analyses point to both the promise and clear constraints of LLM-based translation for complex, literary source material: machine outputs are consistent and sometimes convergent, but accurate interpretative diversity still emerges more strongly among human translators.

We therefore understand fidelity not merely as textual correctness but as the preservation and transformation of ancient conceptual categories. By quantifying where renderings converge (stylistic imprint, genre constraints) and where they diverge (interpretive breadth, cultural framing), we approximate an ``echo'' of the ancient mind, i.e., those durable constraints that persist across modern horizons. The next step is integrative: combine these quantitative maps with targeted close readings and controlled concept inventories to model interpretive distance explicitly. Rather than seeking a single correct translation, we reconstruct how antiquity is being understood: which parts resonate stably and which are reshaped in reception.

%TC:ignore
%\quickwordcount{paper}
%\quickcharcount{paper}
%\detailtexcount{paper}

\section*{Limitations}
This research is limited by the selection of a small number of source texts and translations, which focus solely on Ancient Greek historiography and epic. Evaluation is restricted to English outputs, and the LLMs tested are limited to the leading available models at the time of writing. The alignment assessment was only partially automated and used a subset of translation pairs; broader validation will be required for generalisable conclusions. Finally, automatic evaluation metrics may not fully capture literary subtlety or deep interpretive fidelity.


\section*{Ethical Considerations}
We adhere to the ethical guidelines of both our institution and the conference. All primary texts are in the public domain. Translations used are either public domain or used in limited scope under academic fair use/fair dealing exceptions. No copyrighted material was shared beyond permissible bounds with external systems. No sensitive personal data is present. Automated analyses were transparently documented for reproducibility, and all generated LLM outputs were critically evaluated to avoid propagation of misleading or culturally insensitive material. We further acknowledge the potential for LLM tools to introduce historical or cultural biases and flag this as an ongoing area of concern and scrutiny.

% Print the biblography at the end. Keep this line after the main text of your paper, and before an appendix. 
\printbibliography

% You can include an appendix using the following command
\appendix

\section{Prompts} \label{appdx:first}

\begin{lstlisting}[caption={Prompt generated by Claude to align the Ancient Greek text segments to the translations. The placeholders \texttt{\{xml\_content\}} and \texttt{\{chapter\}} have been replaced with the respective XML parts during batch processing.},label={lst:prompt-to-align}] 
You will be given two inputs: an XML file containing an ancient Greek text split into sentences, and a text file containing an English translation of the same text. Your task is to align the sentences from the ancient Greek text to the corresponding parts of the English translation.
Here is the Greek XML input:

<greek_xml>{xml_content}</greek_xml>

Here is the English translation:

<english_translation>{chapter}</english_translation>.

The alignment can be 1-to-many, i.e., one Greek sentence can correspond to several English sentences. Follow these steps to complete the task:
1. Preprocess the inputs:
  a. Parse the Greek XML to extract individual sentences and their IDs.
  b. Split the English translation into sentences.
2. For each Greek sentence:
  a. Identify the corresponding sentences in the English translation.
  b. Use a combination of the following techniques to find the best matching English sentence(s):
    - Length-based matching (compare the number of words)
    - Key term matching (look for proper nouns, numbers, or other distinctive words)
    - Semantic similarity (use context to determine if the content matches)
  c. If a single Greek sentence aligns with multiple English sentences, group them together.
  d. If multiple Greek sentences align with a single English sentence, keep them grouped with that English sentence.
3. Record the alignments in the following format
   <alignment confidence="1.0">
   <greek_sentence id="[ID from XML]">[Greek sentence]</greek_sentence>
   <english_translation>[Corresponding English sentence(s)]</english_translation>
   </alignment>
4. Here are examples of good and bad alignments:
  Good alignment:
  <alignment confidence="1.0">  <greek_sentence id="1.1"> %\gr{Ἄνδρα μοι  ἔννεπε, Μοῦσα, πολύτροπον, ὃς μάλα πολλὰ}%</greek_sentence> <english_translation>Tell me, O Muse, of the man of many devices, who wandered full many ways</english_translation></alignment>
  Bad alignment:
  <alignment confidence="0.1">  <greek_sentence id="1.2"> %\gr{πλάγχθη, ἐπεὶ Τροίης ἱερὸν πτολίεθρον ἔπερσε}%</greek_sentence>  <english_translation>Tell me, O Muse, of the man of many devices, who wandered full many ways</english_translation></alignment>

Handle these edge cases as follows:
1. If a Greek sentence is split across multiple XML elements, combine them before aligning.
2. If the English translation contains footnotes or explanatory notes, align them separately and mark them as notes:
<note>[Note content]</note>
3. If there are quotes or dialogue in either text, ensure they are aligned properly, maintaining the structure of the conversation.

Provide your alignments in the order they appear in the original Greek text. Be thorough and precise in your alignments, and use the tentative and unaligned tags when necessary. Add a confidence attribute to each alignment, showing how confident you are about the alignment on a scale from 0 to 1. Just output the resulting XML nicely formatted without any additional comments. Process the whole file without asking to proceed with more.
\end{lstlisting}

\begin{lstlisting}[caption={Prompt generated by Claude to align the Ancient Greek text segments to the translations. The placeholder \texttt{\{xml\_content\}} has been replaced with the respective XML parts during batch processing.},label={lst:prompt-to-translate}]
You are tasked with translating an XML document containing Ancient Greek text into English while preserving the original XML structure. Here is the input XML:

<greek_xml>{xml_content}</greek_xml>

Follow these steps to complete the translation:
1. Process each <sentence> element within the XML structure.
2. For each <sentence> element:
  a. Extract the Ancient Greek text.
  b. Translate the text into English to the best of your ability.
  c. Replace the original Greek text with your English translation.
3. Maintain the exact XML structure, including all div elements, their IDs, and nested sentence elements.
4. Preserve any whitespace or formatting present in the original XML.
5. Do not alter, add, or remove any XML tags or attributes.
6. After processing all sentences, output the entire translated XML document, maintaining its original structure.

Begin your response with <translated_xml> and end it with </translated_xml>. Ensure that the content between these tags is a valid XML document with the same structure as the input, but with English translations replacing the Greek text in each <sentence> element.
\end{lstlisting}  

\section{Collocation analysis}
\label{app:second}
These heatmaps show the collocation comparison between translations ( \texttt{PMI} $=$ pointwise mutual information, \texttt{chi\_sq} $=$ chi square).

\begin{figure}[ht!]
    \centering
    \includesvg[width=\linewidth]{images/homer-bigrams.svg}
    \caption{Statistics of bigrams of the Homer translations (Spearman's $\rho$ left, Kendall's $\tau$ right).}
    \label{fig:homer-bigram}
\end{figure}


\begin{figure}[ht!]
    \centering
    \includesvg[width=\linewidth]{images/homger-trigram.svg}
    \caption{Statistics of trigrams of the Homer translations (Spearman's $\rho$ left, Kendall's $\tau$ right).}
    \label{fig:homer-trigram}
\end{figure}


\begin{figure}[ht!]
    \centering
    \includesvg[width=\linewidth]{images/thuc-bigram.svg}
    \caption{Statistics of bigrams of the Thucydides translations (Spearman's $\rho$ left, Kendall's $\tau$ right).}
    \label{fig:thuc-bigram}
\end{figure}

\begin{figure}[ht!]
    \centering
    \includesvg[width=\linewidth]{images/thuc-trigram.svg}
    \caption{Statistics of trigrams of the Thucydides translations (Spearman's $\rho$ left, Kendall's $\tau$ right).}
    \label{fig:thuc-trigram}
\end{figure}


%TC:endignore


\end{document}
