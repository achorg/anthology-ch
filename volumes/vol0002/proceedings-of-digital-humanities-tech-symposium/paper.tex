% THIS IS A LATEX TEMPLATE FILE FOR PAPERS INCLUDED IN THE
% *Anthology of Computers and the Humanities*. ADD THE OPTION
% 'final' WHEN CREATING THE FINAL VERSION OF THE PAPER. 
% DO NOT change the documentclass
\documentclass[final]{anthology-ch} % for the final version
%\documentclass{anthology-ch}         % for the submission

% LOAD LaTeX PACKAGES
\usepackage{booktabs}
\usepackage{graphicx}
% ADD your own packages using \usepackage{}

% TITLE OF THE SUBMISSION
% Change this to the name of your submission
\title{Proceedings of Digital Humanities Tech Symposium 2025: Introduction}


% Add yourself as an author if you contribute. Alpha by last name

% AUTHOR AND AFFILIATION INFORMATION
% For each author, include a new call to the \author command, with
% the numbers in brackets indicating the associated affiliations 
% (next section) and ORCID-ID for each author.  
\author[1]{Rebecca Sutton Koeser}[
  orcid=0000-0002-8762-8057
]


\author[2]{Jose Angel Hernandez}[
  orcid=0000-0002-0969-1106
]

% While we encourage including ORCID-IDs for all authors, you can
% include authors that do not have one by definining an empty ID.

\author[3]{Julia Damerow}[orcid=0000-0002-0874-0092]

% While we encourage including ORCID-IDs for all authors, you can
% include authors that do not have one by definining an empty ID.
\author[4]{Malte Vogl}[
  orcid=0000-0002-2683-6610
]

\author[5]{Robert Casties}[
  orcid=0009-0008-9370-1303
]

% There should be one call to \affiliation for each affiliation of
% the authors. Multiple affiliations can be given to each author
% and an affiliation can be given to multiple authors. 
\affiliation{1}{Center for Digital Humanities, Princeton University, Princeton, NJ, USA}
\affiliation{2}{Research Computing Center, Florida State University, Tallahassee, FL, USA}
\affiliation{3}{School of Complex Adaptive Systems, Arizona State University, Tempe, AZ, USA}
\affiliation{4}{Structural Changes of the Technosphere, Max Planck Institute of Geoanthropology, Jena, Germany}
\affiliation{5}{Digital Humanities Team, Max Planck Institute for History of Science, Berlin, Germany}

% KEYWORDS
% Provide one or more keywords or key phrases seperated by commas
% using the following command
\keywords{digital humanities, proceedings, technical, research software, dhtech}

% METADATA FOR THE PUBLICATION
% This will be filled in when the document is published; the values can
% be kept as their defaults when the file is submitted
\pubyear{2025}
\pubvolume{2}
\pagestart{1}
\pageend{5}
\conferencename{Digital Humanities Tech Symposium 2025}
\conferenceeditors{Julia Damerow and Rebecca Sutton Koeser}
\doi{10.63744/dCDQd0sTPLEk}  

\addbibresource{bibliography.bib}

%%%%%%%%%%%%%%%%%%%%%%%%%%%%%%%%%%%%%%%%%%%%%%%%%%%%%%%%%%%%%%%%%%%%%%%%%%%
% HERE IS THE START OF THE TEXT
\begin{document}

\maketitle

\begin{abstract}
We present and introduce the proceedings from the Digital Humanities Tech Symposium 2025, a one-day workshop organized by DHTech and  held at the Alliance of Digital Humanities Organizations (ADHO) 2025 conference in Lisbon, Portugal. We provide background on DHTech (an ADHO Special Interest Group), discuss the challenges of presenting and publishing on the technical aspects of work in Digital Humanities, and describe the submission and review process for the Symposium presentations and proceedings papers. We conclude with reflections on the challenges of writing about and reviewing technical work such as the papers included in this volume, and our hopes for continuing to improving the spaces and processes for this work in the future.
\end{abstract}

We are pleased to introduce the first-ever published proceedings from a DHTech event, the \textit{Digital Humanities Tech Symposium 2025}, a one-day workshop held at the Alliance of Digital Humanities Organizations (ADHO) 2025 conference in Lisbon, Portugal. Ever since its founding after a workshop at DH2017 in Montreal, DHTech has been active in organizing workshops, mini-conferences and meet-ups to foster community and share knowledge. Once DHTech became an official ADHO Special Interest Group (SIG) in 2021 \cite{dhtech_adho_2021}, we have organized pre-conference workshops every year for the annual ADHO conference.\footnote{For more on the history of DHTech and current efforts, see Julia Damerow's recent blog post celebrating eight years of DHTech. \cite{damerow_lets_2025}.}  We make an effort to document and share the outcome of our events, especially since we know the DHTech community is geographically distributed in ways that make a single meeting time impossible for everyone to join. Our event post-publications range from informal blog posts writing up conversations and resources from virtual meet-ups, to white papers like the one that came out of a workshop at the DH2019 conference in Utrecht \cite{casties_2019_17353345}, to published articles such as the 2024 paper on code review, which expanded on a poster presented by members of the DHTech Code Review Working group at DH2022 \cite{damerow_code_2025, damerow_establishing_2022}.

DHTech is an international grassroots community of people in the Digital Humanities (DH) community whose professional responsibilities involve the development of research software; members include research software engineers, database specialists, computational  researchers, and others.  While there is increasing alignment between the traditional DH developer role and the Research Software Engineer (RSE),  the community intentionally includes people in many different roles and not everyone identifies as developer or RSE. DHTech was established to ``support the development and reuse of software in the Digital Humanities by providing a community to exchange knowledge, share expertise, and foster collaboration'' \cite{dhtech_about_2023}.   Our mission is to establish and improve practices related to developing and maintaining research software, tools, and infrastructure in the digital humanities, topics that often receive limited attention within the broader DH community. We advance this mission through virtual and in-person workshops and events, working groups, and other efforts. We know from a recent survey on Research Software Engineering in DH that the technical community is quite dispersed, and there is no single conference everyone doing technical DH work attends, but that the annual ADHO conference is currently the best option for an in-person gathering  \cite{damerow_surveying_2025}.

In spite of our commitment to presenting and sharing on the technical aspects of Digital Humanities work, it remains difficult to find venues to present and publish on this work. In the models of conference material dissemination listed by the editors of this Anthology, we typically fall between ``none'' and ``self-archiving''  \cite{10.63744@rWDzgqfDLYNm},  usually by posting materials on the \href{https://dh-tech.github.io/}{DHTech website}.\footnote{https://dh-tech.github.io/} We are grateful for the leadership of the Association for Computers and the Humanities (ACH) and editors of the new \textit{Anthology of Computers and the Humanities} for their ``inclusive and generous sense'' of Digital Humanities work \cite{10.63744@rWDzgqfDLYNm} and offering opening this space for us to document and share more widely the work of the technical community represented by DHTech.  We are also encouraged by recent developments in this space, including the \href{https://www.cambridge.org/core/journals/computational-humanities-research/information/author-instructions/preparing-your-materials#softwarepapers}{software paper}\footnote{https://www.cambridge.org/core/journals/computational-humanities-research/information/author-instructions/preparing-your-materials\#softwarepapers } in the \textit{Computational Humanities Research} journal, and the technical track in the call for proposals for the DH2026 conference \cite{noauthor_cfp_nodate}, which expands on the earlier technical review option made available for DH2024 conference submissions \cite{noauthor_call_nodate}.

The papers included in these proceedings are based on presentations included in the \textit{Digital Humanities Tech Symposium}, a one-day workshop organized by DHTech held on July 14th, 2025 as part of the pre-conference program for the DH2025 conference in Lisbon, Portugal. The symposium inverted the typical format of presentations at the DH conference by focusing on the technical aspects of digital humanities projects, while still situating them within their specific research and disciplinary contexts. Presentations ranged from implementation details of research projects, to tool presentations and live demonstrations touching on infrastructure, authentication, and publication platforms. The aim of this one-day conference was to facilitate the exchange of practical technical knowledge that participants could take back and apply to their own work.

To ensure rigor, quality, and technical relevance, we implemented a submission and review process for symposium presentations.  We published a call for presentation proposals as well as a call for reviewers; proposals were submitted as short abstracts, with links to relevant software packages, and were each reviewed by two volunteer reviewers from the DHTech community. Twelve submissions were selected for the symposium based on reviewer feedback and available time slots in the symposium schedule.  We became aware of the new \textit{Anthology of Computers and the Humanities} and the opportunity for published proceedings from our symposium after the submission process was already underway. Thus, we re-publicized the call for proposals with this option, and communicated with all presenters to determine their interest and capacity for expanding their presentations into a short paper to be published in the proceedings. While not all presenters were interested or able to contribute, there was enough interest to warrant publishing the proceedings. The papers included here represent a strong cross-section of the technical work showcased at the symposium, and we look forward to see publications in other venues on the symposium presentations that are not represented here.

With over 60 pre-registrations, the symposium was fully booked and saw strong attendance. We started the day with over 40 attendees, which included walk-in attendees who did not pre-register for the workshop, and we had a peak audience of around 50 participants. Based on an informal Mentimeter poll at the beginning of the symposium, nearly everyone who participated in the poll (26 out of 27) programs as part of their work (all the time or sometimes), and 17 out of 28 came to the symposium for the technical talks. This attendance and engagement demonstrates the strong interest of the DH community in understanding and learning from the technical aspects of DH work.

Symposium presenters who opted to write a paper for the proceedings were encouraged to draft their papers before the symposium, and were given a deadline to submit a final version two weeks after the DH2025 conference. Our goal was to take advantage of the proceedings publication model for ``rapid dissemination and citation of work that has been presented to and reviewed by experts'' \cite{10.63744@rWDzgqfDLYNm}, and to avoid the work becoming outdated due to the fast pace of change of technology. We recruited a second round of reviewers, which included paper authors as well as other DHTech members who attended the symposium, and facilitated a second review process. Each paper was reviewed by two people within a time frame of about two weeks. The Proceedings Committee, which consisted of the members of the DHTech Steering Committee, read through and discussed all reviews. Authors revised their papers in response to reviewer comments and then further refined their papers based on more detailed feedback from the proceedings editors, Damerow and Koeser. It is worth noting that we did not provide specific review guidelines to the reviewers, giving them the freedom to independently determine which aspects of the papers and related code to focus on. Based on our review of the responses, there are important questions that need to be discussed further about how best to write about and review technical work, since DH projects cover such a wide range of humanities disciplines, technologies and methods.  We look forward to improving our process and developing more flexible and rigorous standards as we collaborate with DH2026 conference organizers on the review process for the technical track, as well as for future DHTech publication opportunities.

The symposium was designed from the outset to include different kinds of talks -- some showcased tools, while others presented on technical aspects of specific projects. The presentations varied widely, not only in terms of the type of implementation (from Python packages to multi-tier, distributed architectures), but also in focus; some presentations highlighted practical usage of specific tools, while others detailed the underlying processes and design assumptions built into their systems. We particularly appreciate those presenters who connect their work in a specific context to larger questions or insights; such as, what qualifies as research software, and to what degree research infrastructure ``counts''; how community-driven development with multiple stakeholders can be managed; the need to develop research software in the context of specific projects with expert collaborators; or that taking a new implementation approach for pragmatic reasons can also be a fun technical challenge.

We recognize that the role and process for conference proceedings varies by field, and that papers are typically peer-reviewed in advance of the conference \cite{10.63744@rWDzgqfDLYNm}. We look forward to having a proceedings publication option at the outset when organizing future DHTech events, and will continue to refine guidelines for authors and reviewers that are relevant and helpful for a diverse range of technical work.  As the conference and publication landscape for digital and computational humanities continues to change, we will continue to make space for presenting, publishing, and converations on technical work in DH. 

\section*{Acknowledgements}

We are grateful for the support that ADHO offers to Special Interest Groups such as DHTech, which includes the time, space, and infrastructure for pre-conference events, such as our in-person Symposium at DH2025. We are grateful to the leadership of ACH and the editors of the new \textit{Anthology for Computers and the Humanities}, particularly Taylor Arnold and Lauren Tilton, for the opportunity to publish our proceedings. We thank and acknowledge everyone who contributed to the 2025 Digital Humanities Tech Symposium by submitting, presenting, reviewing, and attending, and we are grateful to all the many, diverse members of the DHTech community.

% Print the biblography at the end. Keep this line after the main text of your paper, and before an appendix. 
\printbibliography

% You can include an appendix using the following command
\appendix
\section*{Symposium and Proceedings Committee}

The symposium was organized and facilitated by the DHTech Steering Committee.  All committee members were involved in both rounds of review.

\begin{itemize}
    \item Robert Casties  *\textbf{\textdagger}\textbf{‡} 
    \item Julia Damerow  *\textbf{\textdagger}\textbf{‡}
    \item Jose Angel Hernandez  *\textbf{\textdagger}\textbf{‡} 
    \item Rebecca Sutton Koeser *\textbf{\textdagger}\textbf{‡} 
    \item Jeffrey Tharsen *\textbf{\textdagger}
    \item Malte Vogl *\textbf{\textdagger}
\end{itemize}

\section*{Reviewers}

\begin{itemize}
    \item Iain Emsley *
    \item Jamie Folsom \textbf{\textdagger}\textbf{‡} 
    \item Paul Girard *\textbf{‡} 
    \item Benjamin Kiessling *\textbf{‡} 
    \item Gregor Middell \textbf{\textdagger}\textbf{‡} 
    \item David Ragnar Nelson \textbf{\textdagger}\textbf{‡}
    \item Diego Siqueira \textbf{\textdagger}
\end{itemize}

*: presentation submission reviewer

\textbf{\textdagger}:  proceedings reviewer

\textbf{‡} : symposium presenter


\end{document}
