% THIS IS A LATEX TEMPLATE FILE FOR PAPERS INCLUDED IN THE
% *Anthology of Computers and the Humanities*. ADD THE OPTION
% 'final' WHEN CREATING THE FINAL VERSION OF THE PAPER. 
% DO NOT change the documentclass
\documentclass[final]{anthology-ch} % for the final version
%\documentclass{anthology-ch}         % for the submission

% LOAD LaTeX PACKAGES
\usepackage{booktabs}
\usepackage{graphicx}
% ADD your own packages using \usepackage{}

% TITLE OF THE SUBMISSION
% Change this to the name of your submission
\title{Introducing the Anthology for Computers and the Humanities}

% AUTHOR AND AFFILIATION INFORMATION
% For each author, include a new call to the \author command, with
% the numbers in brackets indicating the associated affiliations 
% (next section) and ORCID-ID for each author.  
\author[1]{Taylor Arnold}[
  orcid=0000-0003-0576-0669,
  email=tarnold2@richmond.edu,
  corresponding=true
]

\author[2]{Maria Antoniak}[
  orcid=0000-0002-4807-2850
]

\author[3]{Miguel Escobar Varela}[
  orcid=0000-0001-8396-1664
]

\author[4]{Marie Puren}[
  orcid=0000-0001-5452-3913
]

\author[5]{Mila Oiva}[
  orcid=0000-0002-5241-7436
]

\author[6]{Amanda Regan}[
  orcid=0000-0002-4260-5839
]

\author[7]{Lauren Tilton}[
  orcid=0000-0003-4629-8888,
  email=ltilton@richmond.edu
]

\author[8]{Melanie Walsh}[
  orcid=0000-0003-4558-3310
]

% There should be one call to \affiliation for each affiliation of
% the authors. Multiple affiliations can be given to each author
% and an affiliation can be given to multiple authors. 
\affiliation{1}{Data Science and Statistics, University of Richmond, U.S.A.}
\affiliation{2}{Computer Science, University of Colorado Boulder, U.S.A.}
\affiliation{3}{Faculty of Arts and Social Sciences, National University of Singapore}
\affiliation{4}{Laboratoire de Recherche de l'EPITA, Paris, France}
\affiliation{5}{History and Archaeology, University of Turku, Finland}
\affiliation{6}{History and Geography, Clemson University, U.S.A.}
\affiliation{7}{Rhetoric and Communication Studies, University of Richmond, U.S.A.}
\affiliation{8}{Information School, University of Washington, U.S.A.}


% KEYWORDS
% Provide one or more keywords or key phrases seperated by commas
% using the following command
\keywords{digital humanities}

% METADATA FOR THE PUBLICATION
% This will be filled in when the document is published; the values can
% be kept as their defaults when the file is submitted
\pubyear{2025}
\pubvolume{1}
\pagestart{1}
\pageend{5}
%\conferencename{Proceedings of Conference XXX}
%\conferenceeditors{Editor1 Editor2}
\doi{00000/00000}  

\addbibresource{bibliography.bib}

%%%%%%%%%%%%%%%%%%%%%%%%%%%%%%%%%%%%%%%%%%%%%%%%%%%%%%%%%%%%%%%%%%%%%%%%%%%
% HERE IS THE START OF THE TEXT
\begin{document}

\maketitle

\begin{abstract}
We present the \emph{Anthology of Computers and the
Humanities}, a new publication designed to host open-access,
peer-reviewed proceedings from digital humanities (broadly defined)
conferences and workshops. The \emph{Anthology} provides a home for
conference proceedings, offering essential structures such as DOIs and
searchability, while being explicitly attuned to the needs of research
situated in humanities disciplines.
\end{abstract}

\section{Background}

The Anthology for Computers and the Humanities builds on developments
across academia. Papers presented at academic conferences and workshops
are frequently made publicly available in the form of proceedings.
Proceedings can be published as a special issue or section of an
existing journal, as a printed book, or---as is increasingly the case in
many fields---in special-purpose digital publication venues. Proceedings
enable the rapid dissemination and citation of work that has been
presented to and reviewed by experts within a field, both for those
attending the event in person and for others who read the proceedings
after the conference has concluded. Papers within proceedings are
frequently peer-reviewed as part of the conference itself and are
written in formats that can be easily adapted to archival forms, such as
LaTeX converted to PDF or Markdown converted to PDF or HTML. Papers
typically do not go through extensive third-party copyediting. These
features make proceedings ideal for quickly disseminating research in an
open platform.

The role of conference proceedings differs significantly across fields \cite{kochetkov2021importance}.
In many social sciences domains, proceedings are considered precursors
to longer papers in journals, with proceedings treated as working drafts
rather than independent publications \cite{lisee2008conference}. Proceedings from mathematics and
statistics conferences, when produced, are often treated as
supplementary details that contain technical proofs and information that
cannot be included in the presentations themselves. Within some
subfields of philology and linguistics, proceedings are edited,
published, and "counted" as equivalent to chapters in edited volumes. In
computer science, proceedings have taken a particularly prominent role
in their respective fields \cite{vrettas2015conferences}.
Across almost every subfield of computer
science, papers published in conferences have become the de facto form
of publication over the past several decades.

Scholarly associations frequently host open platforms for conference
proceedings, providing access to cutting-edge research before, during,
or right after the conference. The Institute of Electrical and
Electronics Engineers (IEEE), for example, hosts a Conference
Proceedings platform that hosts proceedings dating back to 1936. The
Association for Computing Machinery (ACM) also manages the International
Conference Proceeding Series, which is reported to receive an average of
4.4 million page views and 1.75 million downloads per month. The AIP
Conference Proceedings, published by the American Institute of Physics,
and the Association for Computational Linguistics\textquotesingle{} ACL
Anthology provide similar structures within their respective fields.
Platforms provided by scholarly associations typically offer many of the
same bibliometric tools provided by journals, such as DOIs, a top-level
ISSN, and inclusion in research search aggregator tools
\cite{anderson2012towards}. A limitation of
these platforms hosted by scholarly associations, however, is that their
disciplinary focus may make them less suitable for research situated
outside their core fields.

As an alternative to dedicated platforms for publishing proceedings,
many conferences post PDF files of papers directly on their website or
link to general-purpose archival platforms such as arXiv or
Zenodo.\footnote{In addition to proceedings platforms, it is also become
  common to publish computer science related articles on arXiv as a way
  of both circumventing the paywalls of some sites such as the IEEE
  conference proceedings, in order to get feedback before the final
  conference publication is published, and as a way of attempting to
  establish ownership over an idea ahead of other papers published at a
  similar time.} These options are most popular with fields that treat
proceedings as grey literature, which are distinct from research
published in journal articles and printed books. Publishing directly on
a conference website offers an easy and inexpensive way of making
conference proceedings available without the support of a scholarly
association. However, this comes along with many downsides. The
conference site may only continue to exist for a short time, making
papers only temporarily available, and does not come along with the
kinds of bibliometric tools that are important to making research
findable, and are often essential for research to "count" as published
research for the authors. While augmenting these direct links with
general-purpose archival sites such as Zenodo or arXiv provides some
level of long-term access and persistent identifiers, the fact that
anyone can upload their materials to these sites without any form of
peer review causes challenges for considering such work as formal
research outputs. The Anthology for Computers and the Humanities builds
on the opportunities afforded by models across academia to support
peer-reviewed, open access research published on a timeline that aligns
with computationally-infused research to maximize our
scholarship\textquotesingle s impact \cite{bosman2021oa, suber2012open}.

\section{State of Conference Proceedings in the Digital Humanities}

Digital humanities (DH), a term we intend in its most inclusive and
generous sense, has no broadly accepted stance on the role of conference
proceedings within the field. This is not particularly surprising since
DH brings together scholars and scholarship from across fields ranging
from the humanities to computer science, which, as described above, have
very different norms surrounding the status of conference proceedings as
research. We are aware of at least four different models that
DH-oriented conferences have used to disseminate conference proceedings
materials online:

\begin{enumerate}
\def\labelenumi{\arabic{enumi}.}
\item
  \textbf{None.} Many conferences publish only presentation titles and
  (possibly) abstracts on their websites, avoiding any creation of full
  proceedings. For example, this approach is also taken by the most
  recent installations of the ACH's annual conference.
\item
  \textbf{Self-Archiving}. Other conferences publish submissions
  presented at their conferences on their websites. This is the format
  used, for example, by recent annual ADHO conferences, which creates a
  book of abstracts. The status of these as citable, peer-reviewed
  publications can be unclear.
\item
  \textbf{CS workshops}. Within computer science, many conferences allow
  proposals for co-located workshops. Papers presented at these
  workshops must adhere to the rules of the leading conference and can
  often be published through institutionally backed platforms used for
  the main conference proceedings. Recent examples include the
  LaTeCH-CLfL 2025 workshop, co-located with NAACL, which publishes
  proceedings to the ACL Anthology, and the VIS4DH 2025 workshop,
  co-located with IEEE VIS, which publishes proceedings to \ldots.
\item
  \textbf{Dedicated Volumes in CS Proceedings}. The Computational
  Humanities Research (CHR) conference published complete conference
  proceedings in the CEUR Workshop Proceedings from 2020 through 2024.
\item
  \textbf{Special Issues.} Another format used for some conferences is
  to publish papers within a special issue of a journal. This seems less
  common in recent years because it requires a lot of work for everyone
  involved. Previous examples include the special issues of Digital
  Humanities Quarterly, which published papers from BeNeLux 2017, and
  the Göttingen Dialog in Digital Humanities in 2015 and 2016.
\end{enumerate}

The examples above only include conference proceedings that directly
publish papers as accepted to a given event. Many events subsequently
invite presenters to revise their work for inclusion in related
publications, such as a special issue of \emph{Digital Scholarship in
the Humanities} dedicated to an ADHO conference and issues of the
\emph{DH Benelux Journal}. These publications require additional writing
and peer-review processes and therefore function more like regular
journal articles than conference proceedings.

All the examples cited above are reasonable approaches in many
situations, depending on the needs and goals of the conference
organizers and the communities in which they are situated. For example,
a workshop seeking to engage with practitioners in computer science may
find co-locating with a computer science conference to be an ideal
situation. Likewise, a community that encourages the rapid revision of
papers for a special issue of a focused journal may decide that
publishing conference proceedings is a hindrance to subsequent
publications. Offering a self-archiving option with limited
bibliographic support can, similarly, be a valuable compromise for
making work accessible without limiting the impact of subsequent
publications.

In many cases, however, the options listed above simply represent the
best available alternatives among those currently accessible to the DH
community. Additionally, even in cases where these options may appear
promising, they may not be accessible to everyone. To be included as a
computer science workshop requires a clear computer science focus and
cannot be designed around a geographical community. The Computational
Humanities Research team was recently informed that they would no longer
be able to publish in the CEUR Workshop Proceedings. Finally, the
Digital Humanities Quarterly no longer accepts proposals with a "purely
event-based focus." There is a clear need for another option.

\section{Goals and Structure of the Anthology}

The \emph{Anthology of Computers and the Humanities} provides the DH
community with a platform for publishing conference and workshop
proceedings. We are guided by a strong commitment to broad
accessibility, both in terms of the content we publish and how it is
accessed. The \emph{Anthology} is accessible online with no sign-in
required to access the research and no author contribution fees.

All events of various sizes and from any of the broadly defined areas of
DH are welcome to publish proceedings. Every volume of the
\emph{Anthology} corresponds to the proceedings from a specific event.
The organizers of the event serve as, or select, the volume editors and
will have complete editorial control over the content of the papers
published in their volume. The volume editors are invited, but not
required, to include an introductory article to serve as context for the
volume. Any group can submit proposals for volumes in the anthology,
provided that the papers will undergo a peer-review process and that the
event has an open call for participation. We expect that all proposals
for volumes that meet the stated guidelines will be accepted for
inclusion in the \emph{Anthology}.

We have designed the Anthology to maximize the findability and
accessibility of all papers. Papers appearing in the \emph{Anthology}
will be published under a Creative Commons Attribution 4.0 International
license (CC-BY 4.0), which permits all reuse as long as the authors are
properly cited. Copyright for the paper remains with the authors.
Through an institutional license from the non-profit organization
Crossref, we will be able to offer dedicated DOIs for every paper and
ensure inclusion in all major research aggregators
\cite{chandrakar2006digital, hendricks2020crossref}. Additionally, we
will support integration and linked data using ORCIDs and a top-level
ISSN for the publication. These bibliographic tools enhance the
findability of papers published in the Anthology and facilitate
authors\textquotesingle{} ability to receive academic credit for their
publications through their own national and institutional frameworks.
Because we want papers to be easy to cite, ready-to-use citation fields
are provided on every article page, along with a site-wide,
comprehensive, downloadable BibTeX file containing citations to all
published papers from the \emph{Anthology}. Papers will be made
available as downloadable PDFs for archival purposes and searchable HTML
for findability.

Organizationally, the \emph{Anthology} operates as an independently
managed sub-entity under the Association for Computers and the
Humanities (ACH). This structure follows the standard format of computer
science proceedings published by scholarly associations, providing a
platform to operate immediately within an existing non-profit
organization. ACH has generously provided the funds and infrastructure
that have allowed the \emph{Anthology} to operate without charging any
publication costs to authors or organizations. While the ACH is a
US-based professional society, the \emph{Anthology} encourages
contributions from all geographic regions and welcomes papers and
volumes in any language.

\section{Looking Ahead}

We hope that the \emph{Anthology of Computers and the Humanities} makes
an essential contribution to DH by providing all interested members of
the community with access to an open publication platform, through which
they can share their work. The roles of conference proceedings and even
conferences themselves will, as always, continue to evolve. We are
committed to the long-term preservation of published volumes in the
\emph{Anthology}. We are excited to continue adapting as we aim to help
enable, acknowledge, and amplify the research being done within our
community.

\printbibliography

\end{document}
